%%%%%%%%%%%%%%%%%%%%
%
% DIRECTIONS
% Students count off 1, 2, 3, ..., 7, do the one question,
% then explain their answer to their neighbor.
% Perhaps volunteers fill out answer key (+doc camera).
%
% TO PRINT SOLUTIONS
% Typeset with class option "answers",
% ie, \documentclass[10pt, answers]{exam}
%
%%%%%%%%%%%%%%%%%%%%

\RequirePackage{luatex85}
\documentclass[10pt, answers]{exam}
\usepackage{tikz, pgfplots, amsmath, mathtools, multicol, xargs, setspace}
%\pgfplotsset{compat=1.15}
\usepackage[letterpaper, margin=0.5in, column sep=1in]{geometry}
\setlength\parindent{0mm}

% define macro \Repeat{integer}{content}
\usepackage{expl3}
\ExplSyntaxOn
\cs_new_eq:NN \Repeat \prg_replicate:nn
\ExplSyntaxOff

\usepackage{hyperref}

\setlength\fillinlinelength{0.7in}
\setlength\answerclearance{0.7ex}
\CorrectChoiceEmphasis{\bfseries\color{red}}

% diagram for questions on pages 1 and 3
\def\yscaleone{2.6} % = amount to shrink diagram in y direction
\newcommand\diagram{
\fullwidth{
\vspace{-1em}
\begin{center}
\begin{tikzpicture}
  [very thick, yscale=1/\yscaleone]
  % grid
%  \draw[gray, semithick, ystep=100, xstep=2, xshift=-1cm] (1,0) grid (12,20.5);
  \draw[gray, semithick, xstep=100, ystep=2, xshift=-1cm] (0,0) grid (15.25,20.5);
  % x- and y-axes
  \draw[<->, semithick] (-1,21) -- node[pos=0.03, right] {$y$} (-1,0) -- node[pos=0.99, above] {$x$} (14.5,0);
  % labels on y-axis
  \foreach \y in {0,10,...,200}
  \draw (-1,\y/10) node[left] {\y};
  % labels on x-axis
  \foreach \x in {1,...,12}
  \draw (\x,0) node[below] {\x};
  % vertical asymptotes
  \foreach \x in {8,...,12}
  \draw[dashed] (\x,0) -- (\x,20.5);
  % horizontal asymptotes
  \foreach \y in {9,6}
  \draw[dashed] (-1,\y) -- (14.25,\y);
  % plot five curves on left half
  \draw[smooth] plot coordinates{(-1,8.85) (0.2,8.6) (1,7) (2,2) (3.15,1.15) (4,5)};
  \draw (4,8) -- (5,7);
  \draw (5,16) to [bend right=10] (6,10);
  \draw (6,14) -- (7,13);
  \draw (7,17) to [bend left=10]  (8,11);
  % plot five asymptotic curves on right half
  \draw[smooth] plot coordinates {(8.1,20.5) (8.2,15) (8.8,15) (8.9,20.5)};
  \draw[smooth] plot coordinates {(9.1,20.5) (9.2,12) (9.8,8) (9.9,0)};
  \draw[smooth] plot coordinates {(10.1,20.5) (10.2,13) (10.8,13) (10.9,20.5)};
  \draw[smooth] plot coordinates {(11.1,0) (11.2,10.5) (11.8,10.5) (11.9,0)};
  \draw[smooth] plot coordinates {(12.1,0) (12.4,5) (14.25,5.85)};
  % open circles
  \foreach \x/\y in {2/2, 3/1, 4/5, 4/8, 5/7, 5/16, 6/10, 7/17, 8/11}
  \draw[black, fill=white, yscale=\yscaleone] (\x,\y/\yscaleone) circle (3pt);
  % closed circles
  \foreach \x/\y in {1/7, 3/4, 5/19, 6/14, 7/13, 8/3, 11/15, 12/18}
  \draw[black, fill=black, yscale=\yscaleone] (\x,\y/\yscaleone) circle (2.5pt);
  % f(x) label
  \draw (1.5,15) node[fill,white] {\color{black} \huge$f(x)$};
  \end{tikzpicture}
  \end{center}
  }
  
  \vspace{-1em}
  }


\newcommandx\myfillin[6][1=\relax,2=black,3=white,usedefault]{
  \\[-0.7ex]
  \color{#2}
  {#1 \lim_{x\to\mathrlap{\mathbf{#4}}\phantom{0^+}}{}}
  \smash{#5}
  &\color{#2}=\color{white}
  \rlap{\hspace{0.5\fillinlinelength}\clap{\color{#3}\bfseries\text{#6}}}
  \color{white}
  \fillin[#6]
  }

% template for questions 1 through 12 on page 1
\newcommandx\threelim[4]{
\question $\left\{\begin{aligned}
  \\[-6ex]
  \myfillin[\color{white}]{}{f(\arabic{question})}{#1}
  \myfillin{\arabic{question}^-}{f(x)}{#2}
  \myfillin{\arabic{question}^+}{f(x)}{#3}
  \myfillin{\arabic{question}}{f(x)}{#4}
  \end{aligned}\right.$
  }

% template for questions 15 through 26 on page 2
\newcommandx\fivelim[6][
  1={{grid=none, ticks=none}}, 2=\relax, 3=black, usedefault
  ]{
  \def\displayfx{#4}
  \def\displayfzero{#5}
  \def\axisfunction{#6}
  \def\axisstyleoptions{#1}
  \def\axisoptions{#2}
  \def\colordoublelimit{#3}
  \fivelimcontinued
  }
\newcommandx\fivelimcontinued[9][
  1=\relax, 2=\relax, 3=\relax, 4=\relax, 5=\relax, 6=\relax,
  7=white, 8=white, 9=white, usedefault
  ]{
  \question
  \(\begin{matrix}
  \begin{tikzpicture}
  \begin{axis}[
    every axis/.append style={font=\tiny}, 
    xmin=-3.1, xmax=3.1,
    ymin=-2.4, ymax=2.4,
    samples=99,
    major tick length={0},
    line width=1pt,
    axis lines=center, height=1.6in, width=2in, grid=major,
    restrict y to domain=-2.4:2.4,
    % title={\normalsize{\unboldmath$\displaystyle f(x) = #1$}},
    extra y tick style={y tick label style={right}},
    \axisstyleoptions,
    % samples=59 % speed up compile for testing
    ]
    \addplot [black, smooth, thick, domain=-3:3] {\axisfunction};
    \axisoptions
    \end{axis}
    \end{tikzpicture}
    \end{matrix}\)

  \vspace{-3mm}
  \fullwidth{\(\begin{aligned}
    \myfillin[\color{white}][][#7]{}{\displayfzero}{#1}
    \myfillin[][][#8]{0^-}{\displayfx}{#2}
    \myfillin[][][#8]{0^+}{\displayfx}{#3}
    \myfillin[][\colordoublelimit][#8]{0}{\displayfx}{#4}
    \myfillin[][][#9]{-\infty}{\displayfx}{#5}
    \myfillin[][][#9]{+\infty}{\displayfx}{#6}
    \end{aligned}\)}
  }

\newcommand\opencircle[1]{
  \draw[black, fill=white, thick](axis cs:{#1}) circle [radius=2pt];
  }
\newcommand\closecircle[1]{
  \fill[black] (axis cs:{#1}) circle [radius=2pt];
  }
\newcommand\asymptote[2]{
  \draw[dashed, blue, line width=0.8mm]
  (axis cs: #1) -- (axis cs: #2);
  }

\thispagestyle{empty}
\pagestyle{empty}
\begin{document}
\diagram

\begin{questions}
\section{Find Limits Graphically}
\begin{multicols}3
\threelim{70}{70}{70}{70}
\threelim{undefined}{20}{20}{20}
\threelim{40}{10}{10}{10}
\threelim{undefined}{50}{80}{DNE}
\threelim{190}{70}{160}{DNE}
\threelim{140}{100}{140}{DNE}
\threelim{130}{130}{170}{DNE}
\end{multicols}

\section{Find Limits Involving Infinity Graphically}
\begin{multicols}3
\threelim{30}{110}{$+\infty$}{DNE}
\threelim{undefined}{$+\infty$}{$+\infty$}{$+\infty$}
\threelim{undefined}{$-\infty$}{$+\infty$}{DNE}
\threelim{150}{$+\infty$}{$-\infty$}{DNE}
\threelim{180}{$-\infty$}{$-\infty$}{$-\infty$}
\bigskip
\question $\lim\limits_{x\to \mathbf{-\infty}}f(x) = \color{white}\fillin[90]$
\question $\lim\limits_{x\to \mathbf{+\infty}}f(x) = \color{white}\fillin[60]$
\end{multicols}

\clearpage
\section{Famous Functions}
\setlength\columnsep{8mm}
\raggedcolumns
\begin{multicols}4
\fivelim[][\asymptote{-3,0}{3,0}\asymptote{0,-2}{0,2}]
  {1/x}{1/0}{1/x}
  [undefined][$-\infty$][$+\infty$][DNE][0][0]
\fivelim[][\asymptote{0,-2}{0,2}][white]
  {\ln x}{\ln0}{ln(x)}
  [undefined][DNE][$-\infty$][][DNE][$+\infty$]
\fivelim[samples={501}, xtick={0}, ytick={0}][][white]
  {\sqrt x}{\sqrt0}{sqrt(x)}
  [0][DNE][0][][DNE][$+\infty$]
\fivelim[ytick={-1,1},xtick={0}][\opencircle{0,1}\asymptote{-3,0}{3,0}]
  {\sin(x)/x}{\sin(0)/0}{cos(deg(x*6))*exp(-abs(x/1.25))}
  [undefined][1][1][1][0][0]
\end{multicols}

\vfill
\begin{multicols}4
\fivelim[][\asymptote{-3,0}{3,0}\asymptote{0,-2}{0,2}]
  {1/x^2}{1/0^2}{1/abs(x)}
  [undefined][$+\infty$][$+\infty$][$+\infty$][0][0]
\fivelim[ytick={1}, xtick={0}][asymptote{-3,0}{3,0}]
  {e^x}{e^0}{e^x}
  [1][1][1][1][0][$+\infty$]
\fivelim[][{\addplot[black, smooth, thick, domain=-3:3] (x^3,x);}]
  {\sqrt[3]x}{\sqrt[3]0}{0}
  [0][0][0][0][$-\infty$][$+\infty$]
\fivelim{\cos x}{\cos 0}{cos(deg(x)*2)}
  [1][1][1][1][DNE][DNE][][][red]
\end{multicols}

\vfill
\begin{multicols}4
\fivelim{|x|}{|0|}{abs(x)}
  [0][0][0][0][$+\infty$][$+\infty$]
\fivelim[ytick={1}, extra y ticks={-1}, xtick={0}, samples={501}]
  [\opencircle{0,1} \opencircle{0,-1}]
  {|x|/x}{|0|/0}{floor(x/10)*2+1}
  [undefined][$-1$][1][DNE][$-1$][1]
\fivelim[ytick={-1,1},xtick={0},samples={720}][\asymptote{-3,0}{3,0}]
  {\sin(1/x)}{\sin(1/0)}{sin(deg(1/x*1.5))}
  [undefined][DNE][DNE][DNE][0][0][red][red]
\fivelim[ytick={-2,-1,1,2},xtick={-2,-1,1,2}][
  {\draw[black,thick] (axis cs:-2,-2) edge[-] (axis cs:-1,-2);}
  {\draw[black,thick] (axis cs:-1,-1) edge[-] (axis cs:0,-1);}
  {\draw[black,thick] (axis cs:0,0) edge[-] (axis cs:1,0);}
  {\draw[black,thick] (axis cs:1,1) edge[-] (axis cs:2,1);}
  {\draw[black,thick] (axis cs:2,2) edge[-] (axis cs:3,2);}
  \closecircle{-2,-2} \closecircle{-1,-1} \closecircle{0,0}
  \closecircle{1,1} \closecircle{2,2} \opencircle{-1,-2}
  \opencircle{0,-1} \opencircle{1,0} \opencircle{2,1}
  ]{\lfloor x\rfloor}{\lfloor0\rfloor}{0}[0][$-1$][0][DNE][$-\infty$][$+\infty$]

%%% OMIT, BUT PRETTY: x*sin(1/x)
%\fivelim[ytick={-1,1},xtick={0},samples={720}][\opencircle{0,0}]{x\sin(1/x)}{0\sin(1/0)}{x/6*sin(deg(6/x)) * 0.7*(6-1.2*abs(x))}[undefined][0][0][0][1][1]
\end{multicols}

\newpage
\setstretch{0.1}
\diagram
\vspace{-4mm}
\setlength\columnsep{0.5in}
\section{Identify Infinite, Jump, Removable Discontinuities Graphically}

\vspace{-1mm}
\setlength\fillinlinelength{0.8in}
\question We say $f$
  \rlap{is \fillin[continuous][1in] \textbf{at} $\pmb{x=a}$ if}%
  \phantom{has a \fillin[\textbf{removable}][0.8in] \textbf{discontinuity at} $\pmb{x=a}$ if}
  ${f(a) ={}} \lim\limits_{\mathclap{x\to a^-}}f(x) = \lim\limits_{\mathclap{x\to a^+}}f(x)$
  and all three exist and are finite.

\question We say $f$
  has a \fillin[\textbf{removable}] \textbf{discontinuity at} $\pmb{x=a}$ if
  ${f(a) \ne{}} \lim\limits_{\mathclap{x\to a^-}}f(x) = \lim\limits_{\mathclap{x\to a^+}}f(x)$
  and the last two exist and are \rlap{finite.}

\question We say $f$
  has a \fillin[\textbf{jump}] \textbf{discontinuity at} $\pmb{x=a}$ if
  $\phantom{f(a)={}}\lim\limits_{\mathclap{x\to a^-}}f(x) \ne \lim\limits_{\mathclap{x\to a^+}}f(x)$
  and both exist and are finite.

\question We say $f$
  has a \fillin[\textbf{infinite}] \textbf{discontinuity at} $\pmb{x=a}$ if
  $\phantom{f(a)={}}\lim\limits_{\mathclap{x\to a^-}}f(x)$ or $ \lim\limits_{\mathclap{x\to a^+}}f(x)$
  is infinite and both exist.

\setlength\fillinlinelength{1.5in}
\question The function $f$ above is continuous (cts) at the following integers between 1 and 12: \fillin[1].
\question The function $f$ above has removable discontinuities at the following integers: \fillin[2, 3].
\question The function $f$ above has jump discontinuities at the following integers: \fillin[4, 5, 6, 7].
\question The function $f$ above has infinite discontinuities at the following integers: \fillin[8, 9, 10, 11, 12].


\section{Continuity on an Interval}
\vspace{-4mm}
\question We say $f$ is 
  \textbf{continuous on the open interval
  \boldmath $(a,b)$}
  if $f$ is continuous at every $x$ 
  in $(a,b)$.
  \par Find the union of all open intervals
  \textbf{on which $f$
  is continuous}.
  \\[3ex] (Set-builder notation) \fillin[\smash{\raisebox{2mm}{\pmb{$\{x\mid x\ne 2,3,4,\dots,12\}$}}}][5in]
  \\[3ex] (Interval notation) \fillin[\smash{\raisebox{2mm}{\pmb{$
  (-\infty,2)\cup(2,3)\cup(3,4)\cup\dots\cup(11,12)\cup(12,\infty)
  $}}}][5in]
\question We say $f$ is \textbf{continuous everywhere}
  if $f$ is continuous at every $x$ in \fillin[\pmb{$(-\infty,\infty)$}].
\question We say $f$ is 
  \textbf{a continuous function} 
  if $f$ is continuous at every
  $x$ in its \fillin[domain].

\vspace{-1mm}
\section{Left and Right Continuity}
\vspace{-4mm}

\question We say $f$
  is \fillin[left continuous] \textbf{at} $\pmb{x=a}$ if
  ${f(a) ={}} \smash{\lim\limits_{\mathclap{x\to a^-}}}f(x)$
  and both exist and are finite.
  
\question We say $f$
  is \fillin[right continuous] \textbf{at} $\pmb{x=a}$ if
  ${f(a) ={}} \smash{\lim\limits_{\mathclap{x\to a^+}}}f(x)$
  and both exist and are finite.

\question The function $f$ above is left continuous at the following integers between 1 and 12: \fillin[1, 7].
\question The function $f$ above is right continuous at the following integers between 1 and 12: \fillin[1, 6].

\question We say $f$ is \textbf{continuous on the closed interval \boldmath $[a,b]$}
  if $f$ is continuous at every $x$ in the \emph{open} interval $(a,b)$
  \\[1ex]
  and is \fillin[right continuous][2in] at $x=a$
  and is \fillin[left continuous][2in] at $x=b$.

\question The function $f$ above is continuous on the closed interval [ \fillin[6][5mm],\fillin[7][5mm] ] with integer endpoints between 1 and 12.
%
\end{questions}
\end{document}