\RequirePackage{luatex85}
\documentclass{article}
\usepackage{tikz, amsmath, mathtools, multicol, setspace}
\usepackage[margin=0.5in, column sep=1in]{geometry}
\setlength\parindent{0mm}
\thispagestyle{empty}
\pagestyle{empty}
%\everymath{\displaystyle}

% define macro \Repeat{integer}{content}
\usepackage{expl3}
\ExplSyntaxOn
\cs_new_eq:NN \Repeat \prg_replicate:nn
\ExplSyntaxOff

\newcommand\blob[2][15mm]{\(\begin{matrix}\begin{tikzpicture}
	\node[text width=#1, align=flush center]{\begin{spacing}{0.7}#2\end{spacing}};
	\end{tikzpicture}\end{matrix}\)}
\newcommand\myoverset[2]{\smash{\large$\overset
	{#1}{\vphantom{\bigg(}\hspace{#2mm}}
	$}}

\newcommand\diagram[3]{
\begin{tikzpicture}[very thick, scale=2.1, yscale=0.85]
	#3
	% axes
	\draw[<->] (0,2.75) node[below left] {distance [mi]} to (0,0) to (5.75,0) node[right] {time [hr]};
%	% crop
%	\clip (0,0) rectangle (8.5,2.5);
	% f(x)
	\draw[domain=0.5:5.65]
		plot (\x, {(\x-1)^4*5/8/100+15/100}) node [right] {\,\,\large$#1$};
	% secant
	\draw[red, domain=2.5:6] 
		plot (\x, {0.75*(\x-3)+0.25})
		node [right] {\,\,\large secant};
	% tangent
	\draw[red, domain=1.25:6] plot (\x, {0.2*(\x-3)+0.25})
		node [right] {\,\,\large tangent};
		#2
	\end{tikzpicture}}

\begin{document}
\large

\Repeat{2}{
% margin=0.5in, row sep=1in, so textheight=4.5in
\begin{minipage}[t][4.5in]{\textwidth}
1. \textbf{Definition.} The Limit Definition of Derivative

%\vspace{-1ex}
Suppose the distance a train has traveled is a function of time $y=f(x)$.

\vspace{-2ex}
\hfil
\diagram{y=f(x)\phantom{{}=\frac58(x-1)^4+15}}{
	% tick marks and coordinates for closed circles
	\foreach \x/\n in {3/x, 5/x+h}{
		\draw (\x, {(\x-1)^4*5/8/100+15/100}) coordinate (\x);
		\draw (\x, 3pt) -- ++(down:6pt) node[below] {\large$\n\vphantom h$};
		}
	% closed circles (yscale counteracts global yscale)
	\foreach \x in {3, 5}{
		\draw[black, fill=black, yscale=1/0.85] (\x) circle (1.5pt);
		}
	% fillin (x1, y1) = (x, f(x))
	\draw (3) node [above left] {
		\bfseries\huge(\myoverset{x_1}{14},\myoverset{y_1}{14})\!
		};
	% fillin (x2, y2) = (x+h, f(x+h))
	\draw (5) node [above left] {
		\bfseries\huge(\myoverset{x_2}{18},\myoverset{y_2}{18})\!
		};
	}{
	}

\vfill
\blob{avera\smash ge velocity over $[x, x{+}h]$}
= \blob{avera\smash ge rate of change over $[x, x{+}h]$}
= \blob{slope of secant between $x, x{+}h$}
=\hfill$\frac{\Delta y}{\Delta x}$\hfill%
=\hfill$\frac{y_2 - y_1}{x_2 - x_1}$\hfill%
=\hfill{\LARGE\boxed{\phantom{\frac{f(x+h)-f(x)}h}}}\hfill%
= \blob[22mm]{definition of difference quotient}

\vfill
\blob{velocity at $x$}
= \blob{rate of change at $x$}
= \blob{slope of tangent at $x$}
=\hfill{\LARGE\boxed{\lim_{h\to\quad}\phantom{\frac{f(x+h)-f(x)}h}}}\hfill%
=\hfill\!definition of derivative $f'(x)$

\vfill
\end{minipage}

\vspace{1in}
}

\Repeat{2}{
\begin{minipage}[t][4.5in]{\textwidth}
%\clearpage
\begin{tabbing}
2. \textbf{Example.} \= Use the definition of difference quotient to find the slope of  \\ \> the \textbf{secant} line through $y=f(x)=\frac58(x-1)^4+15$ between $x$ and $x+h$. \\ \> \emph{Hint:} $f(x+h)$ \== $\scriptstyle\frac58((x-1)+h)^4+15$ \\ \>\>= $\scriptstyle\frac58(x-1)^4+4(x-1)^3h+6(x-1)^2h^2+4(x-1)h^3+4h^4+15$\\[1ex]
3. \textbf{Exercise.} \> Use the definition of derivative to find the slope of \\ \>  the \textbf{tangent} line to $y=f(x)=\frac58(x-1)^4+15$ at $x$. \\[-2.5ex]
\diagram{y=f(x)=\frac58(x-1)^4+15}{
	% tick marks and coordinates for closed circles
	\foreach \x in {0, ..., 5}{
		\draw (\x, {(\x-1)^4*5/8/100+15/100}) coordinate (\x);
		\draw (\x, 3pt) -- ++(down:6pt) node[below] {\large$\x$};
		}
	% closed circles (yscale counteracts global yscale)
	\foreach \x in {3, 5}{
		\draw[black, fill=black, yscale=1/0.85] (\x) circle (1.5pt);
		}
	% y-tick marks
	\foreach \y in {0, 100, 200}{
		\draw (3pt, \y/100) -- ++(left:6pt) node[left] {\large$\y$};
		}
	}{
	% grid
	\draw[gray, semithick, dotted, step=0.25] (0,0) grid (5.75,2.75);
	\draw[gray, thick] (0,0) grid (5.75,2.75);
	}\\[1ex]
4. \textbf{Exercise.} \> Use the \textbf{diagram} to estimate the slope of  \\ \> the \textbf{secant} line to $y=f(x)=\frac58(x-1)^4+15$ \textbf{between 3 and 5}. \\[1ex]
5. \textbf{Exercise.} \> Use your answer to Exercise 2 above to estimate the slope of  \\ \> the \textbf{tangent} line to $y=f(x)=\frac58(x-1)^4+15$ \textbf{at 3}.
\end{tabbing}
\end{minipage}

\vspace{1in}
}
\end{document}